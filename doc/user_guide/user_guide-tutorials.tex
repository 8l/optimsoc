\chapter{Tutorials}
\label{chap:tutorials}

The best way to get started with OpTiMSoC after you've prepared your
system as described in the previous chapter is to follow some of our
tutorials. They are written with two goals in mind: to introduce some
of the basic concepts and nomenclature of manycore SoC, and to show
you how those are implemented and can be used in OpTiMSoC.

Some of the tutorials (especially the first ones) build on top of each
other, so it's recommended to do them in order. Simply stop if you
think you know enough to implement your own ideas!

\section{RTL Simulation: Compute Tile and Software}

It is a good starting point to simulate a single compute tile of a
distributed memory system. Therefore a simple testbench is included
and demonstrates the general simulation approach and gives an insight
in the software building process.

Simulating only a single compute tile is essentially an OpenRISC core
plus memory and the network adapter, where all I/O of the network
adapter is not functional in this test case. It can therefore only be
used to simulate local software.

You can find this example in \verb|tbench/rtl/dm/compute_tile|. You
need to have Modelsim installed before you run:

\begin{verbatim}
> make build
\end{verbatim}

The output should be as follows:

\begin{verbatim}
Model Technology ModelSim SE-64 vlog 10.1b Compiler 2012.04 Apr 26 2012
-- Compiling module tb_compute_tile
-- Compiling module trace_monitor
[..]
-- Compiling module lisnoc_mp_simple
-- Compiling module lisnoc_mp_simple_wb

Top level modules:
	tb_compute_tile
\end{verbatim}

In case you see errors check that the environment variables are set
correctly.

\medskip
The testbench can now be started in the Modelsim user interface or can
be executed in command line mode using \verb|make sim|, but there is
no software loaded to the memory, what results in a warning:

\begin{verbatim}
# ** Warning: (vsim-7) Failed to open readmem file "ct.vmem" in read mode.
\end{verbatim}

The simulations always expect vmem files that initialize the memories.
This needs to be generated from the compiled source code. Before you
build your own software you will need the support libraries. You can
find all system software and example codes in
\verb|src/sw/system/dm/|.

In the system software folder you will find the \verb|libbaremetal|
library. Go to the \verb|build| directory of the library and
\verb|make| it. The code should compile without errors and warnings.

The library contains many necessary functions, the boot code and
helper macros. Two versions are build: \verb|libbaremetal.a| is the
normal distributed memory baremetal software and
\verb|libbaremetal-paging.a| is required when the local tile memory
needs to be initialized from global memory, e.g., on an FPGA. This
will later be extended to support smaller tile memories and do page
swapping etc. This partially works, but an alternative approach to
initialization and paging is described below with the PGAS system.

When you link the library to your own code, you need to link some
external symbols as described in \verb|sysconfig.h|. Those are
definitions specific to a certain system instance, such as the
dimensions, the presence of I/O, the distribution of compute
ressources, etc.

Therefore you will always need to link your system-specific
\verb|optimsoc-sysconfig.o| to the library and your application
objects, what we will describe in the following. The
\verb|optimsoc-sysconfig.c| can be found as part of the compute tile
testbench (\verb|tbench/rtl/dm/compute_tile|) under \verb|sw|.

To build a simple ``Hello World'' example, simply switch to \verb|sw|.
The example application can be found in the software path. You should
copy the whole path

\begin{verbatim}
$> optimsoc-app-link dm baremetal/hello_simple
\end{verbatim}

Inside the \verb|hello_simple| folder you can find the
\verb|hello_simple.c| and the \verb|Makefile|. Now \verb|make| the
example. This will automatically also build the sysconfig.o and link
all together to the elf file. Furthermore some other files are build:

\begin{itemize}
\item \verb|hello_world.dis| is the disassembly of the file
\item \verb|hello_world.bin| is the elf file als loaded binary file
\item \verb|hello_world.vmem| is a textual copy of the binary file
\end{itemize}

(Note: If the latter is not build, ensure you have \verb|bin2vmem| in
your \verb|PATH|, see
\ref{chap:installation}).

You can now run the example. First go back to the compute tile
testbench main folder. Before the simulation warned that
\verb|ct.vmem| cannot be found. Therefore we simply link the software
to this filename

\begin{verbatim}
> ln -s sw/hello_simple/hello_simple.vmem ct.vmem
\end{verbatim}

If you now run the software (\verb|make sim|), the simulation should
terminate with:

\begin{verbatim}
# [               52500, 0] Software reset
# [            57480000, 0] Terminated at address 0x00008080
\end{verbatim}

\verb|0x00008080| is the program counter (can vary depending on your cross
compiler version) the simulation terminated. This is correct behavior
and will be explained below. Furthermore you will find a file called
\verb|stdout| which shows the actual output:

\begin{verbatim}
[            51185000, 0] Hello World!
\end{verbatim}

The \verb|[..]| part is the time stamp and core id. But how does the
actual printf-output get there when there is no UART or similar?

OpTiMSoC software (especially in future releases) makes excessive use
of a tricky part of the OpenRISC ISA. The ``no operation'' instruction
\verb|l.nop| has a paramter \verb|K| in assembly. This can be used for
simulation purposes. It can be used for instrumentation, tracing or
special purposes as writing characters with minimal intrusion or
simulation termination.

The termination is forced with \verb|l.nop 0x1|. If you have a look at
the disassembly \verb|hello_simple.dis| at instruction \verb|0x8080|
(see above) you exactly find this instruction. The actual action is
then done with the trace monitor.

With this method you can simply provide constants to your simulation
environments. For variables this method is extended by putting data in
further registers (often \verb|r3|). This still is minimally intrusive
and allows you to trace values. The printf is also done that way (see
newlib):

\begin{verbatim}
void sim_putc(unsigned char c) {
  asm("l.addi\tr3,%0,0": :"r" (c));
  asm("l.nop %0": :"K" (NOP_PUTC));
}
\end{verbatim}

This function is called from printf as write function. The trace
monitor captures theses characters and puts them to the stdout file.

You can easily add your own ``traces'' using a macro defined in
\verb|optimsoc.h|:

\begin{verbatim}
#define OPTIMSOC_TRACE(id,v)                \
   asm("l.addi\tr3,%0,0": :"r" (v) : "r3"); \
   asm("l.nop %0": :"K" (id));
\end{verbatim}

This feature is used extensively by current and future OpTiMSoC software.

\section{Going Multicore: Simulate Small 2x2 Distributed Memory
  System}

Next you might want to build an actual multicore system. You can find
such a system in \verb|tbench/rtl/dm| directory as
\verb|system_2x2_cccc|. The nomenclature in all pre-packed systems
first denotes the dimensions and then the instantiated tiles, here
\verb|cccc| as four compute tiles.

When you switch to this directory, you can build the system using
\verb|make|. In the following build the \verb|hello_simple| software
identical to the description above. After running \verb|make sim| you
will find the files \verb|stdout.0| to \verb|stdout.3|, each
containing ``Hello World''.

There is a second ``hello world'' example available:
\verb|hello_simplemp|. This program uses the simple message passing
facilities of the network adapter to send messages. In that example
all cores send a message to core 0, that prints a message when
receiving the packets. When all cores sent their messags, the core
acknowledges this by printing its own ``Hello World'' message.

The simulation should output:

\begin{verbatim}
# [               52500, 0] Software reset
# [               52500, 1] Software reset
# [               52500, 2] Software reset
# [               52500, 3] Software reset
# [            46585000, 1] Event 0x0100: 0x00000000
# [            46617500, 1] Event 0x0100: 0x00000001
# [            46627500, 2] Event 0x0100: 0x00000000
# [            46652500, 1] Event 0x0100: 0x000fffec
# [            46660000, 2] Event 0x0100: 0x00000001
# [            46670000, 3] Event 0x0100: 0x00000000
# [            46695000, 2] Event 0x0100: 0x000fffec
# [            46702500, 3] Event 0x0100: 0x00000001
# [            46737500, 3] Event 0x0100: 0x000fffec
# [            47272500, 0] Exception # 8 occured
# [            48292500, 1] Terminated at address 0x0000dd5c
# [            48335000, 2] Terminated at address 0x0000dd5c
# [            48377500, 3] Terminated at address 0x0000dd5c
# [            48835000, 0] Event 0x0102: 0x00000001
# [            48867500, 0] Event 0x0102: 0x00000000
# [            48895000, 0] Event 0x0102: 0x00000001
# [            65855000, 0] Event 0x0103: 0x00000000
# [            66337500, 0] Event 0x0102: 0x00000002
# [            66352500, 0] Event 0x0102: 0x00000000
# [            66360000, 0] Event 0x0102: 0x00000001
# [            75940000, 0] Event 0x0103: 0x00000000
# [            76420000, 0] Event 0x0102: 0x00000003
# [            76435000, 0] Event 0x0102: 0x00000000
# [            76442500, 0] Event 0x0102: 0x00000001
# [            85920000, 0] Event 0x0103: 0x00000000
# [           100387500, 0] Terminated at address 0x0000dd5c
\end{verbatim}

After each core is reset they start their execution and after a
certain time emit events. We will have a more detailed view on the
events in the remaining tutorial. For the moment, you should know that
the \verb|0x100| events describe that the processors send their
messages. \verb|Exception # 8| is the interrupt exception, that occurs
when data arrives at the network adapter. Core 0 then receives all of
the messages (event \verb|0x102|).

Finally, a real world example is given by \verb|heat|. Link this
example similar as above (do not forget to link the correct ct.vmem at
all time!). This example calculates the heat distribution in a
distributed manner. The cores coordinate their boundary regions by
sending messages around. With the debug infrastructure below the trace
events will be much easier to understand.

\section{The Look Inside: Introducing the Debug System}

In the previous tutorials you have seen some software running on a simple
OpTiMSoC system. Until now, you have only seen the output of the applications,
not how it works on the inside.

This problem is one of the major problems in embedded systems: you cannot
easily look inside (especially as soon as you run on real hardware as opposed
to simulation). In more technical terms, the system's observability is
limited. A common way to overcome this is to add a debug and diagnosis
infrastructure to the SoC which transfers data from the system to the outside
world, usually to a PC of a developer.

OpTiMSoC also comes with an extensive debug system. In this section, we'll have
a look at this system, how it works and how you can use it to debug your
applications. But before diving into the details, we'll have a short discussion
of the basics which are necesssary to understand the system.

Many developers know debugging from their daily work. Most of the time it
involves running a program inside a debugger like GDB or Microsoft Visual
Studio, setting a breakpoint at the right line of code, and stepping through the
program from there on, running one instruction (or one line of code) at a time.

This technique is what we call run-control debugging. While it works great for
single-threaded programs, it cannot easily be applied to debugging parallel
software running on possibly heterogenous many-core SoC. Instead, our solution
is solely based on tracing, i.e. collecting information from the system while
it is running and then being able to look at this data later to figure out the
root cause of a problem.

The debug system consists of two main parts: the hardware part runs on the
OpTiMSoC system itself and collects all data. The other part runs on a
developer's PC (often also called host PC) and controls the debugging process
and displays the collected data. Both parts are connected using either a USB
connection (e.g. when running on the ZTEX boards), or a TCP connection (when
running OpTiMSoC in simulations).

\section{Verilator: Compiled Verilog simulation}

At the moment running ``verilated'' simulations is the best supported
way of observing the system traces. We will therefore run the examples
from before using a verilated simulation and observing the system in
the graphical user interface.

In the following we will have a look at building such a system and how
to observe it with the GUI. In \verb|tbench/verilator/dm| you find
systems identical to the RTL simulation. We will directly start with
the \verb|system_2x2_cccc|. In the base folder you should simply make
it:

\begin{alltt}
\$> make
\end{alltt}

The command will first generate the verilated version of the 2x2
system. Finally it builds the toplevel files and links to
\verb|tb_system_2x2_cccc| and \verb|tb_system_2x2_cccc-vcd|. The
latter generates a full VCD trace file of the hardware, which is
much slower and also easily takes up tens of GB.

Similar to the steps described above you will need to build the
software, e.g., the heat example. Again you need to link the
\verb|vmem| file. Now start the simulation:

\begin{alltt}
\$> ./tb_system_2x2_cccc
\end{alltt}

It will start a debug server and wait for connections:

\begin{alltt}

             SystemC 2.3.0-ASI --- Feb 11 2013 12:54:17
        Copyright (c) 1996-2012 by all Contributors,
        ALL RIGHTS RESERVED

Listening on port 22000
\end{alltt}

In another console now start the OpTiMSoC GUI:

\begin{alltt}
\$> optimsocgui
\end{alltt}

In the first dialog window you need to set the debug backend to
\emph{Simulation TCP Interface} and proceed then. After the GUI
started you need to connect using \emph{Target
  System}$\rightarrow$\emph{Connect}. The system view should change to
a 2x2 system.

The last step is to run the system by \emph{Target
  System}$\rightarrow$\emph{Start CPUs}. The execution trace on the
bottom of the window will start showing execution sections and events.
By moving the mouse over the section you will find the description of
the section. Similarly for the events you find a short description of
the event.

% \section{PGAS System}

% problem: memory größe

% pgas system bauen und simulieren

% \section{UART I/O}

% problem: auf dem FPGA kein report-printf

% andere sysconfig

% UART tile

% \section{XUPV5 Simulation}

% Funktionale simulation

% \section{XUPV5 Synthesis}

% Synthese auf XUPV5, 2x2\_cmuc

% Erklären wie Speicher initialisieren und bauen
