\chapter{Toolchain}
\label{chap:toolchain}

\section{Build Baremetal Toolchain}

As low level toolchain we will use the \verb|or1k-elf-*| toolchain. At
the moment there are many different versions of this toolchain.

You can either install this toolchain as described in the user guide
or build manually as described here.

The configuration will go to \verb|PREFIX|, which is here set to

\begin{alltt}
\$> export PREFIX=/opt/optimsoc/toolchain
\end{alltt}

You should set this to your preferred path.

Futhermore it is necessary to add the future installation to the
\verb|PATH| variable:

\begin{alltt}
\$> export PATH=\$PREFIX/bin:\$PATH
\end{alltt}

A line like this is necessary in your \verb|.bashrc| or similar to use
the toolchain in the future and incrementally during the build
process.

Next you should go to a temporary build directory and checkout
the sourceware (binutils, gdb, newlib, ..) and the gcc git
repositories:

\begin{alltt}
\$> mkdir ~/tmp-builddir
\$> cd ~/tmp-builddir
\$> git clone https://github.com/TUM-LIS/optimsoc-sourceware.git
\$> git clone https://github.com/TUM-LIS/optimsoc-gcc.git
\end{alltt}

%We recommend to switch to the current (or another release) and not use
%the current development head. You can switch to the latest release by:
%
%\begin{alltt}
%\$> cd optimsoc-sourceware
%\$> git checkout -b rel-current
%\$> cd ..
%\$> cd optimsoc-gcc
%\$> git checkout -b rel-current
%\$> cd ..
%\end{alltt}
%
%Of course you can take any other release, the master branch or other
%branches. It is recommended to use the latest release.

The actual build contains 4 stages as the base binutils are needed to
build gcc and gcc is needed to build the newlib then. Finally the
complete gcc is built.

\subsection{Build Sourceware -- Stage 1}

\begin{verbatim}
$> mkdir bld-sourceware-stage1
$> cd bld-sourceware-stage1
$> ../optimsoc-sourceware/configure --target=or1k-elf --prefix=$PREFIX \
    --enable-shared --disable-itcl --disable-tk --disable-tcl \
    --disable-winsup --disable-gdbtk --disable-libgui --disable-rda \
    --disable-sid --disable-sim --disable-gdb --with-sysroot \
    --disable-newlib --disable-libgloss
$> make
$> make install
$> cd ..
\end{verbatim}

You may run \verb|make -j8| to run 8 parallel build jobs or similar to
speed up the build here and in the following.

\subsection{Build GCC -- Stage 1}

\begin{verbatim}
$> mkdir bld-gcc-stage1
$> cd bld-gcc-stage1
$> ../optimsoc-gcc/configure --target=or1k-elf --prefix=$PREFIX \
    --enable-languages=c --disable-shared --disable-libssp
$> make
$> make install
$> cd ..
\end{verbatim}

\subsection{Build Sourceware -- Stage 2}

\begin{verbatim}
$> mkdir bld-sourceware-stage2
$> cd bld-sourceware-stage2
$> ../optimsoc-sourceware/configure --target=or1k-elf --prefix=$PREFIX \
    --enable-shared --disable-itcl --disable-tk --disable-tcl \
    --disable-winsup --disable-gdbtk --disable-libgui --disable-rda \
    --disable-sid --enable-sim --disable-or1ksim --enable-gdb \
    --with-sysroot --enable-newlib --enable-libgloss
$> make
$> make install
$> cd ..
\end{verbatim}

\subsection{Build GCC -- Stage 2}

\begin{verbatim}
$> mkdir bld-gcc-stage2
$> cd bld-gcc-stage2
$> ../optimsoc-gcc/configure --target=or1k-elf --prefix=$PREFIX \
    --enable-languages=c,c++ --disable-shared --disable-libssp \
    --with-newlib
$> make
$> make install
$> cd ..
\end{verbatim}

