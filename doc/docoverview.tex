% This file is included by the other documents and gives an overview
% about the available documentation.

\section*{OpTiMSoC Documentation Overview}

The OpTiMSOC documentation is organized in four different categories:

\begin{description}
\item[User Guide] The User Guide describes the general usage of the
  OpTiMSoC elements in a tutorial style. It covers the basic building
  processes and how to get stuff running. The User Guide is related to
  releases and distributed via the website and can be built in the
  repository. 
\item[API documentation] The software components are documented using
  Doxygen\footnote{\url{http://www.doxygen.org}}. The generated API
  documentations serve the users when programming software for
  OpTiMSoC. The API documentation is also related to releases and can
  be automatically generated in the repository and are also
  distributed via the website. At the moment the following API
  documentation is available:
  \begin{itemize}
  \item OpTiMSoC Newlib API
  \item OpTiMSoC System Library API
  \item OpTiMSoC Host Software API
  \item OpTiMSoC SystemC Library
  \end{itemize}
\item[Reference Manual] The Reference Manual covers all topics in
  detail. It gives a better insight in how OpTiMSoC is organized and
  how things work. It primarily serves developers as source of
  information when extending OpTiMSoC.
\item[Technical Reports] The Technical Reports are released separately
  and cover implementation details. They therefore serve as
  documentation of components and source of information for
  developers. They do not only cover technical details but most
  importantly also present \emph{why} something works as it does.
\end{description}